\chapter{Introduction}

\section{Motivation}

Rapid digitalisation has transformed how organisations collect and use personal data to deliver services, optimise operations, and support decision-making \parencite{zul2022}.
In Singapore, this shift is amplified by national drives toward Smart Nation goals and widespread AI adoption \parencite{choudhury2024}.
As data use increases, so do risks associated with misuse, unauthorised access, and large-scale breaches.
Recent cases illustrate these risks.
For example, Marina Bay Sands was fined SGD 315{,}000 after a breach affecting more than 665{,}000 customer records, demonstrating that even mature organisations face difficulties in maintaining strong data protection practices \parencite{chan2025}.
To safeguard individuals while supporting innovation, the PDPA was introduced to set baseline obligations for all organisations handling personal data \parencite{pdpc2023}.
However, compliance remains challenging.
PDPA guidelines are extensive, spread across multiple documents, and written for readers with legal or technical expertise.
As regulations evolve, many organisations---particularly SMEs---struggle to interpret requirements correctly.
This difficulty is evident in common operational errors, such as the Credit Counselling Singapore incident where email addresses were exposed due to incorrect use of the ``To'' field \parencite{pdpc2017ccs}.
Such examples highlight the need for accessible and scenario-specific compliance guidance.

\section{Problem Statement}

Although resources produced by the Personal Data Protection Commission (PDPC) are publicly available, they are not always accessible to non-specialists.
Many SMEs lack legal support and rely on ad-hoc interpretations or outdated practices.
Existing commercial compliance tools are often costly or too complex, while general-purpose AI chatbots may hallucinate, mix regulatory frameworks, or offer unsafe advice.
This project seeks to address these gaps by developing AI4PDPA, a modular, AI-assisted compliance platform for Singaporean SMEs.
Its objectives are threefold.
Firstly, AI4PDPA aims to provide accurate, PDPA-grounded, and scenario-specific guidance through a controlled AI chatbot.
Secondly, AI4PDPA aims to streamline compliance workflows using structured document templates and PDPA training modules.
Thirdly, AI4PDPA aims to support AsiaCloud's long-term commercialisation plans with a scalable, cost-efficient architecture and aggregated usage analytics.

\section{Scope}

The project spans two academic terms.
Term 7 focuses on foundational work, which includes defining the overall system architecture, developing an initial chatbot prototype, conducting preliminary testing, and refining project requirements through iterative feedback sessions with the industry partner.
This early phase establishes the technical direction, key constraints, and safety considerations necessary to support a scalable PDPA compliance platform.
Term 8 prioritises full implementation across the platform's core modules.
Planned work includes completing the chatbot's features---such as safe-answer behaviour, PDPA-specific grounding, and more reliable retrieval---together with developing a server-based chat history system to support analytics and longer-term storage.
The team will also implement PDPA document template generation, build a training module that allows administrators to upload training content and create quizzes, and develop an administrative analytics dashboard that presents aggregated usage insights for AsiaCloud.
The system's architecture will emphasise modularity, cost efficiency, and extensibility so that AsiaCloud can build upon this proof-of-concept and scale it into a commercial product after the project concludes.

\section{Stakeholders}

The platform is designed for users who interact with PDPA obligations in day-to-day SME operations.
SMEs form over 99\% of Singapore's enterprises \parencite{lim2025}, and many have limited resources for formal compliance, making accessible PDPA support especially important.
Operational staff---including administrative, customer service, sales, and human resources personnel---handle personal data frequently but often rely on informal or low-digitisation workflows such as spreadsheets or email \parencite{navarrete2019}.
These users require quick clarification when performing tasks such as collecting customer details, responding to access requests, or drafting consent notices.
The chatbot and document templates are designed to support these immediate operational needs.
Managers and internal data stewards typically assume PDPA responsibilities in addition to their primary roles.
They require clearer guidance for interpreting PDPA obligations, onboarding staff, and ensuring consistent handling of personal data across the organisation.
This group benefits from structured training modules and aggregated analytics that highlight recurring compliance gaps.
SME owners and leadership teams use higher-level insights to evaluate organisational compliance readiness.
They rely on the dashboard to understand frequently asked PDPA questions, training participation, and potential areas of risk that require process changes or additional staff support.
Beyond end users, AsiaCloud Solutions is a key stakeholder as both industry partner and future maintainer.
Their interests include evaluating user behaviours, validating market demand, and identifying features suitable for a commercialised product.
The platform's modular architecture and server-based chat logging were designed to support this longer-term vision.
While not within the immediate scope of the project, the architecture allows for future expansion to industry associations, training providers, and larger organisations that require lightweight PDPA reference tools.

\section{Success Criteria}

The platform's success is evaluated across four dimensions: accuracy, safety, usefulness, and scalability.
Accuracy refers to providing PDPA-aligned responses grounded in official guidelines and enforcement cases.
Safety focuses on refusal behaviour for sensitive or out-of-scope queries and on reducing hallucinations through retrieval-based grounding.
Usefulness concerns the practicality of the chatbot, document templates, training modules, and analytics in supporting common SME compliance tasks.
Scalability relates to the system's modular, cost-efficient architecture, ensuring suitability for SMEs and future commercial expansion by AsiaCloud.
These criteria will be assessed through controlled testing, benchmarking, and stakeholder feedback in Term 8.

\section{Constraints}

The project operates under several constraints.
Regulatory constraints require strict alignment with PDPA and limit guidance to publicly available PDPC materials.
Data constraints restrict testing to synthetic or anonymised cases, as real organisational data cannot be used.
Ethical and safety constraints include avoiding hallucinations, avoiding inappropriate advice, and discouraging users from entering sensitive personal data.
Architectural constraints require a lightweight, cost-efficient, and modular design suitable for SMEs and scalable for AsiaCloud.
Finally, evaluation constraints limit testing to controlled environments, as large-scale deployment studies fall outside the scope of the project.
