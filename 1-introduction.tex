\chapter{Introduction}

Rapid digitization has transformed how organizations collect, analyze, and rely on personal data to deliver services, optimize resources, and make critical decisions \parencite{zul2022}.

\lipsum[1-3]

% In Singapore, this shift is significant due to nationwide efforts to build a Smart nation and accelerate AI adoption across industries (Choudhury, 2024).
% As organizations rely heavily on data to drive productivity and innovation, the protection of data has grown into a critical governance concern. At the same time, the risk associated with data misuse, unauthorized access, and large-scale breaches have increased.
% High profile incidents in Singapore demonstrate the tangible consequences of poor data governance. For example, Marina Bay Sands (MBS) was fined S\$315,000 by the Personal Data Protection Commission (PDPC) after a data breach exposed the personal information of more than 665,000 customers, highlighting the magnitude and impact of failures in data protection practices (Chan, 2025). Such cases illustrate that even larger organizations with substantial resources can face compliance challenges, while smaller organizations are often less equipped to manage these risks.
% To ensure responsible data practices, Singapore enacted the Personal Data Protection Act 2012 (PDPA). The PDPA establishes baseline data protection obligations, balancing individual rights with business innovation (PDPA Overview, n.d.). The PDPC further supports compliance through advisory guidelines and compliance toolkits that are publicly available on its website (Advisory Guidelines on Key Concepts in the Personal Data Protection Act, n.d.).
% However, as technology evolves and new regulatory requirements emerge, organizations face ongoing difficulty in interpreting and applying PDPA requirements.
