\documentclass{class}

\title{-}
\author{-}
\date{-}

\renewcommand{\bibname}{References}
\DefineBibliographyStrings{english}{bibliography = {References}}

\begin{document}

\pagenumbering{roman}

\begin{titlepage}
	\thispagestyle{empty}
	{\centering
		\includegraphics[width=0.5\textwidth]{images/sutd-logo.png}

		\vspace{8pt}

		{\large Singapore University of Technology and Design}

		Capstone Term 7

		Capstone 9 Project 07

		\vspace{32pt}

		\begin{center}
			\vbox{%
				\hrule height 0.5pt
				\vskip 6pt
				\hbox to \dimexpr\textwidth\relax{%
					\hfil\textbf{\large AsiaCloud\_AI4PDPA}\hfil
				}%
				\vskip 6pt
				\hrule height 0.5pt
			}%
		\end{center}
	}
	\noindent
	\begin{minipage}[t]{0.5\textwidth}
		\raggedright
		\textit{Author}\\
		Ang Ching Xuen\\
		Fan Xiangwei\\
		Gregory Lim Eu Rhen\\
		Issac Anand Rajaram\\
		Matthew Andrei Salatin Purba\\
		Qi Hengchang\\
		Sherman Kho Jun Hui
	\end{minipage}
	\begin{minipage}[t]{0.5\textwidth}
		\raggedleft
		\textit{Student ID}\\
		1006976\\
		1005533\\
		1007485\\
		1007208\\
		1007094\\
		1007166\\
		1006890
	\end{minipage}
	\vfill
	\begin{center}
		\today
	\end{center}
\end{titlepage}

\chapter*{Acknowledgements}

\lipsum[1-1]

\chapter*{Executive Summary}

Rapid digitalisation in Singapore has increased the volume of personal data handled by organisations and heightened the risks of misuse, unauthorised access, and data breaches.
While the Personal Data Protection Act (PDPA) provides the regulatory framework governing responsible data practices, many organisations---especially small and medium enterprises (SMEs)---continue to struggle with interpreting and applying these obligations in day-to-day operations.
Existing compliance resources are fragmented, technically complex, and often costly, creating a gap for accessible, practical guidance.

To address this challenge, this project collaborates with AsiaCloud to develop AI4PDPA, a modular, AI-driven platform designed to help SMEs understand and apply PDPA requirements more effectively.
The platform incorporates a PDPA chatbot powered by retrieval-augmented generation (RAG), a document template generator, a PDPA training module with automated scoring, and a dashboard that presents aggregated compliance insights.
Together, these components aim to provide real-time, contextualised assistance without requiring legal expertise or expensive consultancy services.

In Term 7, the team established the system architecture, conducted literature analysis, identified ethical and PDPA compliance requirements, developed three iterative prototypes of the chatbot.
Key technical progress included implementing the chunking and storage pipeline, deploying the system on Amazon Web Services (AWS) using containerised services, developing initial front-end interfaces, benchmarking large language models (LLMs) to compare their accuracy and reliability, and integrating early safety and refusal mechanisms aligned with PDPA obligations.

Looking ahead, Term 8 will complete the full platform build.
Priorities enhancing the chatbot experience, integrating server-based chat history, completing the document-generation and training modules, and building an analytics dashboard that provides value for both SMEs and AsiaCloud.
To support long-term scalability, the team will transition from a manual deployment process to a GitHub Actions-based Continuous Integration and Continuous Delivery (CI/CD) pipeline.
The new workflow will automate testing, container builds, and deployment to AWS, reducing operational overhead and ensuring that only validated builds reach production—aligned with modern best practices for development operations (DevOps).
When fully developed, the platform aims to provide SMEs with a low-barrier, affordable, and practical solution for PDPA compliance, while also giving AsiaCloud a strong foundation for a future commercial offering.
By combining AI technologies with regulatory grounding and ethical safeguards, the project seeks to enhance organisational readiness, reduce compliance risks, and contribute to Singapore's broader Smart Nation and data protection objectives.


\tableofcontents

\clearpage

\pagenumbering{arabic}

\chapter{Introduction}

Rapid digitization has transformed how organizations collect, analyze, and rely on personal data to deliver services, optimize resources, and make critical decisions \parencite{zul2022}.
In Singapore, this shift is significant due to nationwide efforts to build a Smart nation and accelerate AI adoption across industries \parencite{choudhury2024}.
As organizations rely heavily on data to drive productivity and innovation, the protection of data has grown into a critical governance concern.
At the same time, the risk associated with data misuse, unauthorized access, and large-scale breaches have increased.
High profile incidents in Singapore demonstrate the tangible consequences of poor data governance.
For example, Marina Bay Sands was fined SGD 315000 by the Personal Data Protection Commission (PDPC) after a data breach exposed the personal information of more than 665000 customers, highlighting the magnitude and impact of failures in data protection practices \parencite{chan2025}.
Such cases illustrate that even larger organizations with substantial resources can face compliance challenges, while smaller organizations are often less equipped to manage these risks.
To ensure responsible data practices, Singapore enacted the Personal Data Protection Act 2012 (PDPA).
It establishes baseline data protection obligations, balancing individual rights with business innovation \parencite{pdpc2023}.
The PDPC further supports compliance through advisory guidelines and compliance toolkits that are publicly available on its website \parencite{pdpc2022}.
However, as technology evolves and new regulatory requirements emerge, organizations face ongoing difficulty in interpreting and applying PDPA requirements.
An example from Singapore illustrates how easily organizations can violate the PDPA during routine operations.
In the case of Credit Counselling Singapore \parencite{pdpc2017ccs}, an employee sent a follow-up email to 96 clients but mistakenly placed their emails addresses in the ``To'' field instead of using ``Bcc''.
As a result, 96 email addresses and associated names were visible to all recipients \parencite{pdpc2017ccs}.
This incident demonstrates that even a simple administrative error can lead to a data breach, regulatory consequences, and the exposure of sensitive personal information.
It also highlights a broader issue: many organizations, especially smaller ones, lack accessible and practical guide to help them understand and comply with PDPA requirements in everyday situations.

% problem statement and objectives
While PDPA information is publicly available, it remains complex and difficult to navigate for non-specialists.
Existing compliance tools may be costly, generalized, or not tailored to Singapore's regulatory context. SMEs, which often operate without dedicated legal or compliance teams, face difficulty obtaining reliable PDPA guidance that is both comprehensible and operationally relevant.
This project addresses the problem by developing an AI-driven platform that provides contextualized PDPA information, practical templates, training materials, and insights into organizational compliance concerns.
The goal is to create a modular, scalable proof-of-concept that meets SME needs while supporting AsiaCloud's longer-term plan to develop a commercial PDPA compliance solution.
The platform aims to integrate a PDPA chatbot, a document template generator, a training module with automated scoring, and a dashboard offering aggregated analytics, all built on a lightweight, cost-efficient architecture suitable for future expansion.

% scope
The project spans two academic terms.
Term 7 focuses on laying the foundation, which includes defining the system architecture, developing a preliminary chatbot prototype, conducting initial testing, and refining the project requirements based on client feedback.
This foundational work clarifies key constraints, safety considerations, and the design direction needed to support the full platform.
Term 8 will focus on implementation.
The planned work includes completing the chatbot's features---such as safe-answer behaviour and Singapore-specific responses---building a browser-based chat history system, generating PDPA document templates, creating a scoring-enabled training module, and developing a dashboard that presents aggregated usage insights.
The dashboard will incorporate both a free version and placeholders for future paid-tier features. The architecture will prioritize modularity, cost efficiency, and extensibility to support AsiaCloud's long-term commercial ambitions.

% stakeholders
The platform is intended primarily for SMEs operating in Singapore, particularly those without dedicated compliance teams.
It is designed to support staff who handle personal data in day-to-day operations, new employees undergoing PDPA training, and managers responsible for data protection practices.
AsiaCloud is also a key stakeholder, as the dashboard's aggregated insights will inform future product development and market positioning.

% success criteria
The success of the platform will be assessed based on its functional accuracy, relevance, and clarity in delivering PDPA information to users. It should demonstrate safe and responsible behaviour when responding to sensitive or out-of-scope queries, while maintaining an intuitive user experience across the chatbot, templates, dashboard, and training modules. The architecture should remain modular, with components that can be independently enabled to support a free version and future paid-tier enhancements. Cost efficiency is also a key criterion, as the platform must remain accessible and sustainable for SMEs. This working definition will be refined further in consultation with the course instructor.

\chapter{Literature Review}

\lipsum[1-5]

% \paragraph{Personal Data Protection in Singapore.}

% Rapid digitalization has increased the scale at which organizations collect and process personal data, making data protection a key concern, especially in Singapore (Zul, 2022). The growing reliance on technologies and artificial intelligence results in the need for a clear governance framework to manage the collection, use, and disposal of personal data responsibly (Schubert and Barrett, 2024). As digital services expand across different industries, the risk of data misuse, unauthorized access and breaches has also grown (KPMG, 2023).

% In response to these challenges, PDPA was established to safeguard individual's personal data while supporting organizations' business interest (PDPA Overview, n.d.). To aid compliance, PDPC uploads advisory guidelines to aid organizations and individuals in their understanding of PDPA (Advisory Guidelines on Key Concepts in the Personal Data Protection Act, n.d.).

% However, despite the availability of these resources, many organizations and individuals struggle to understand and comply with PDPA requirements correctly due to constant evolving regulations (Key Challenges in Achieving PDPA Compliance in 2024, 2024). According to the PDPC's 2015 industry survey, about 58\% of organizations required support to achieve compliance, reflecting knowledge and resource gaps (Industry Survey on the Personal Data Protection Act September 2015, 2015). This suggests that as innovative technologies and regulations emerge, traditional resources may not be sufficient for compliance support. Also, there is growing interest in leveraging artificial intelligence (AI) to automate retrieval of knowledge and comprehension of regulatory compliance (Gültekin-Várkonyi*, 2025). Hence, there is an opportunity to combine AI and compliance to create tools that can interpret legal frameworks and make regulatory knowledge more accessible.

% \paragraph{Compliance Challenges for Small and Medium Enterprises.}

% Small and medium enterprises (SMEs) form the backbone of Singapore economy, accounting over 99\% of all enterprises and employing 70\% of the workforce (Lim, 2025). However, SMEs often face greater obstacles in meeting compliance requirements compared to larger organizations due to limited finance resources and the growing complexity of regulatory obligations (Bello, Idemudia, and Iyelolu, 2024).

% Because SMEs must prioritize day-to-day operations and revenue generation, compliance is viewed as a secondary priority. Hence, many do not allocate sufficient time, budget, or staff to interpret and implement compliance regulations (Common Regulatory Compliance Challenges for SMEs, 2025).
% Digital capability gaps further worsen the compliance challenge. According to a survey conducted by Capterra, 32\% of SMEs still rely on spreadsheets to manage customer's information, while another 35\% uses manual methods or email communication, which are insufficient under modern data protection guidelines (Navarrete, 2019). These informal or decentralized data management methods make it difficult to track consent, update records accurately, or ensure secure retention and disposal.

% In this context, there is a clear need for accessible, low-barrier compliance support tools tailored to SMEs. Solutions such as AI-drive PDPA chatbots can lower the knowledge and resource barrier by providing SMEs with immediate, accurate guidance without requiring legal expertise, formal training, or expensive consultancy services.

% \paragraph{Existing PDPA Compliance Tools and Their Limitations.}

% A variety of resources are available to support PDPA compliance, primarily provided by PDPC. These include advisory guidelines, compliance checklists, assessment tools, and the Data Protection Essentials programmer for SMEs (Kick-starting Your Data Protection Journey, n.d.). Although these materials are comprehensive, they are spread across multiple documents and written in technically dense language, making them challenging non-experts to navigate efficiently. During the team's review of these materials, it became clear that users often struggle to identify which specific sections apply to their situation or to interpret PDPA requirements without legal or technical expertise (A.M. Lonzetta and Hayajneh, 2020).

% In addition to PDPC materials, several commercial and professional solutions are available. Many organizations engage outsourced Data Protection Officers or legal consultants to interpret PDPA requirements on their behalf. Others adopt enterprise-grade privacy management systems such as Varonis or Verasafe, which offer data governance dashboards and risk assessment modules. However, they are costly, require ongoing subscription fees which are impractical for SMEs with limited budgets. For example, Onspring's privacy management software costs approximately \$30 000 to \$56,000 for initial setup with annual subscription fees ranging from \$10 000 to \$50 000 (Randall, 2025). Furthermore, existing solutions lack conversational, real-time interaction. Users must manually search through documentation, navigate dashboards, or rely on external consultants.

% Taken together, these findings highlight the absence of an accessible, affordable, and PDPA-specific digital tool capable of providing real-time compliance support. This gap underscores the need for an AI-powered PDPA chatbot that allows users to obtain accurate, scenario-specific guidance without requiring legal expertise or substantial financial investment.

% \paragraph{AI and Chatbots for Legal and Compliance Support.}

% The increasing complexity of legal and regulatory frameworks has driven organizations to adopt AI solutions to automate and streamline compliance processes (Bleach, 2024). AI technologies such as natural language processing (NLP) and large language models (LLMs) allow systems to interpret, analyze, and generate human-like responses to text-based queries (Vaniukov, 2024). In the legal industry, AI tools are being used for document analysis, legal advice support, and contract drafting, reducing mundane works and improving efficiency (What Is Legal Artificial Intelligence (AI) And How Will It Affect The Next Generation Of Legal Professionals?, n.d.).

% A key innovation in this area is AI chatbots which act as conversational agents to answer user queries about legal and compliance matters. In the legal industry, AI chatbots have already begun transforming how professionals access and interpret information. For instance, Harvey AI, developed on OpenAI's GPT technology, has partnered with law firms and consulting giants such as PwC to assist in legal research, contract review, and compliance analysis (PwC announces strategic alliance with Harvey, positioning PwC's Legal Business Solutions at the forefront of legal generative AI, 2023). This growing adoption highlights the potential for AI to streamline legal workflows and enhance user understanding of complex legal texts.

% Similarly, there is growing recognition that such tools can extend to the data protection and compliance domain. Developments in this area align with Singapore's Smart Nation initiatives, which encourages the use of AI to improve governance, productivity, and security (AI for the Public Good For Singapore and the World, 2023). Therefore, AI-powered PDPA chatbots represent a promising approach to bridging compliance gaps, especially resource-constrained SMEs.

% \paragraph{Privacy and Data Protection Concerns in AI Chatbots.}

% AI chatbots introduce a range of privacy and data protection concerns, particularly when deployed in compliance-related environments. Large Language Models (LLMs) rely on vast amounts of training data and probabilistic generation techniques, which means they may inadvertently produce inaccurate, misleading, or hallucinated responses (Gültekin-Várkonyi, 2025). In legal and regulatory contexts, such errors carry heightened risks, as organisations may unknowingly act on incorrect information. Studies have shown that users often ascribe unwarranted authority to AI systems, further amplifying the potential for misinterpretation (Alkamli et al., 2024).

% A significant privacy risk lies in how chatbots process, store, or transmit user inputs. If poorly designed, chatbots may inadvertently retain personal information through server logs, analytics tools, or model telemetry, creating compliance obligations under data protection regimes such as the PDPA (PDPA Overview, n.d.). Research has highlighted that users frequently input sensitive identifiers—names, national IDs, contact details—into AI systems without understanding how the data will be used or stored (Sebastian, 2023). Such behaviour can expose organisations to liability, especially when combined with insufficiently transparent data handling practices.

% Another concern is the lack of jurisdictional awareness in generic AI models. Without explicit controls, chatbots may provide privacy guidance derived from other countries' regulatory frameworks, leading to inaccurate PDPA interpretations (Osborne Clarke, 2025). This issue was also observed in broader AI deployment studies, which noted that models trained on globally diverse datasets often generate advice incompatible with local regulatory requirements (KPMG, 2023).

% Furthermore, AI chatbots may amplify existing privacy risks by enabling large-scale, automated dissemination of erroneous information. As Marr (2025) notes, the conversational fluency of AI systems can mask underlying inaccuracies, making hallucinations particularly dangerous in compliance settings. These concerns reinforce the need for strict data minimisation, embedded refusal behaviour, jurisdiction-specific guardrails, and robust safeguards to ensure that AI chatbots do not themselves create new vectors of privacy or data protection risk.

% \paragraph{Retrieval-Augmented Generation.}

% Retrieval-Augmented Generation (RAG) has emerged as a key method for improving the factual accuracy, grounding, and explainability of LLM-based systems. Unlike standalone generative models, which rely solely on their pre-trained internal parameters, RAG enhances responses by retrieving relevant information from an external knowledge base and injecting it into the model's context window. This significantly reduces the likelihood of hallucination and supports transparency by grounding answers in verifiable sources (Hillebrand, 2025; Gültekin-Várkonyi, 2025).

% In regulatory domains, RAG is particularly advantageous. It ensures that generated responses are anchored in authorised documents such as PDPC guidelines, enforcement decisions, and advisory notes. This is critical in compliance settings, where organisations must rely on accurate, consistent, and up-to-date interpretations of legal obligations (Chasandras, 2025). RAG also supports modular and iterative updates to the system's knowledge base, allowing new or revised PDPC guidelines to be integrated without retraining the underlying model (Malali, 2025). This flexibility is essential in fast-evolving regulatory landscapes where interpretations and enforcement practices change over time.

% RAG also helps enforce jurisdictional boundaries. By constraining the retrieval corpus to Singapore-specific privacy documents, the system can avoid providing irrelevant or incorrect advice drawn from foreign regulatory frameworks—a frequent problem in non-RAG chatbot deployments (Osborne Clarke, 2025). This targeted retrieval thereby reduces compliance risks associated with general-purpose LLMs that lack built-in awareness of local laws.

% From a privacy perspective, RAG also supports controlled knowledge exposure. Since the retrieved documents are predefined and auditable, their use mitigates concerns regarding reliance on opaque or unpredictably trained model parameters. This aligns well with PDPA's emphasis on accuracy, minimization, and accountability.

% Overall, RAG provides a technologically robust foundation for AI-assisted PDPA guidance, combining the generative flexibility of LLMs with the factual reliability and explainability required in compliance-critical domains.

\chapter{System Design}

To support our goal of helping Singaporean SMEs comply with the PDPA, we introduce a novel chatbot system that aims to answer PDPA-related queries in factual and transparent way.

\begin{figure}[H]
	\centering
	\includegraphics[width=0.75\textwidth]{images/pipeline.png}
	\caption{Pipeline representation of system.}
\end{figure}

\vspace{-12pt}

We design a generation pipeline consisting of three steps: moderation, retrieval, and generation.
Given a user query, the moderation step checks if a query is safe to handle, the retrieval step finds information relevant to the query, and the generation step crafts a response to the query.

On a high level, our system consists of a frontend, backend, and infrastructure.
The frontend handles the exposed logic, the backend handles the non-exposed logic, while the infrastructure hosts the backend and frontend.

\section{Backend Design}

\begin{figure}[H]
	\centering
	\includegraphics[width=0.8\textwidth]{images/backend-1.png}
	\caption{Graphical representation of backend.}
\end{figure}

\vspace{-12pt}

To ensure factuality, the retrieval step only retrieves information from documents defined as truthful.
To ensure transparency, the generation step names each source from which information is retrieved.

Given a set of documents, the system performs the following steps offline.

\begin{enumerate}
	\item \textbf{Chunking.} The system divides each document into smaller pieces called ``chunks''. This is performed to make downstream information consumption more manageable.
	\item \textbf{Embedding.} The system maps each chunk to a numerical representation capturing the semantics of the chunk, called an ``embedding''. This is performed to allow for downstream semantic comparisons with user queries.
	\item \textbf{Indexing.} The system stores each chunk and its corresponding embedding in a database optimised for embedding retrieval. This is performed to allow for efficient downstream retrieval of relevant chunks for a user query.
\end{enumerate}

Given a user query, the system performs the following steps online.

\begin{enumerate}
	\item \textbf{Moderation.} The system checks if the query seems safe to handle.
	If it seems unsafe, the pipeline returns a generic response (e.g. ``Sorry, I cannot help with that.'') to the user; otherwise, the pipeline moves to the next step.
	This is performed to protect the user and our system from any adverse effects that might result from sending an unsafe query to chat model.
	\item \textbf{Retrieval.} The system finds the most relevant chunks to the query by computing a similarity score between the query and each chunk, and returns the highest-scoring chunks.
	This is performed to retrieve relevant information for the query.
	\item \textbf{Generation.} The system queries the chat model with the conversation, query, and chunks, and returns the response of the model.
\end{enumerate}

We had to make some design decisions for our vector database, inference provider, chunking algorithm, embedding model, vector search algorithm, and chat model.

\begin{table}[H]
	\centering
	\begin{tabular}{l@{\qquad}lll}
		\toprule
		& \textbf{Chroma} & \textbf{Pinecone} & \textbf{Qdrant} \\
		\midrule
		\textbf{License class} & Open & Closed & Open \\
		\textbf{GitHub stars} & ~25k & ~3k & ~27k \\
		\bottomrule
	\end{tabular}
	\caption{Comparison between Chroma, Pinecone, and Qdrant.}
\end{table}

\vspace{-12pt}

We found 3 vector database abstractions from our research: Chroma, Pinecone, and Qdrant.
Ultimately, we settled on Qdrant as the open license class gives greater developmental control, and the larger community support---as measured by the number of GitHub stars---facilitates debugging.

\begin{table}[H]
	\centering
	\begin{tabular}{l@{\qquad}lll}
		\toprule
		& \textbf{GroqCloud} & \textbf{OpenRouter} & \textbf{Vertex AI} \\
		\midrule
		\textbf{Relative model count} & Smallest & Largest & In-between \\
		\textbf{Embedding model support} & Non-existent & Existent & Existent \\
		\bottomrule
	\end{tabular}
	\caption{Comparison between GroqCloud, OpenRouter, and Vertex AI.}
\end{table}

\vspace{-12pt}

We found 3 inference provider abstractions: GroqCloud, OpenRouter, and Vertex AI.
Ultimately, we settled on OpenRouter as its model support provides developmental flexibility.

\lipsum[1-3]

\section{Frontend Design}

\lipsum[1-3]

\section{Infrastructure Design}

\lipsum[1-3]

\chapter{System Implementation}

\section{Term 7}

\subsection{Iteration 1}

\lipsum[1-3]

\subsection{Iteration 2}

\lipsum[1-3]

\section{Term 8}

\lipsum[1-3]

\chapter{System Validation}

\section{System Verification}

\lipsum[1-3]

\section{System Evaluation}

\lipsum[1-3]

\chapter{Conclusion}

\section{Achievements}

Term 7 focused on establishing the foundation for a safe, accurate, and scalable PDPA compliance assistant.
We validated the problem through literature review, market analysis, and discussions with AsiaCloud, confirming that SMEs face fragmented resources, limited expertise, and a lack of affordable PDPA tools.
This validation informed the design of the core system architecture and the selection of the frontend, backend, and cloud technologies.

An initial chatbot prototype was developed with basic retrieval capabilities, allowing PDPA-aligned responses grounded in official PDPC documents.
Two iterations improved UI design, retrieval quality, safety behaviour, and early features such as conversation storage, PDF export, and refusal logic.
Benchmarking of three LLM candidates was conducted using a preliminary RAG pipeline to reduce hallucination rates.

From an engineering standpoint, Term 7 included setting up the cloud deployment pathway, comparing AWS compute options, and implementing a basic continuous integration workflow.
The team also drafted the structure and data flow for the training platform.
These efforts provided the technical clarity and feasibility checks required for full implementation in Term 8.

\section{Insights}

First, we found that retrieval-based grounding is essential for accuracy.
Early tests showed that even strong models occasionally mixed PDPA with foreign regulations such as the GDPR, underscoring the need for a tightly scoped retrieval corpus and jurisdiction-specific prompting.
This guided both our model-selection criteria and the structure of the RAG pipeline.

Second, discussions with AsiaCloud highlighted the importance of modularity for future commercialisation.
Each component---chatbot, document templates, training modules, and analytics---must operate independently so AsiaCloud can scale or monetise them selectively.

Risk assessment exercises also reinforced the need for clear safeguards, including refusal behaviour, input sanitisation, and strict data minimisation aligned with PDPA requirements.
These considerations influenced prompting strategy, guardrails, and the move toward controlled server-side storage.

Finally, AWS exploration showed that managed compute services such as App Runner offer a good balance of performance, scalability, and operational simplicity.
This became the foundation for the Term 8 deployment plan.

\section{Budget Summary}

\begin{table}[H]
	\centering
	\begin{tabular}{l@{\qquad}lll}
		\toprule
		& \textbf{Unit cost} & \textbf{Units} & \textbf{Total cost} \\
		\midrule
		\textbf{Transportation} & SGD 30/cab & 16 cabs & SGD 480 \\
		\textbf{Backend} & SGD 15/month & 7 months & SGD 105 \\
		\textbf{Infrastructure} & SGD 108.75/month & 7 months & SGD 756 \\
		\bottomrule
	\end{tabular}
	\caption{Project budget summary.}
\end{table}

\vspacebaselineskip

The project's operating costs remain intentionally low to ensure that the platform is viable for SME adoption and long-term sustainability.
Based on the current architecture, AWS infrastructure costs amount to approximately SGD 108 per month, covering App Runner compute, storage, load balancing, and operational monitoring.
LLM usage through OpenRouter adds an estimated SGD 15 per month, bringing the total estimated monthly cost to roughly SGD 124 per month.
One-time expenditures were minimal, with transport and logistics forming the primary non-technical cost.
Overall, the cost profile aligns with AsiaCloud's requirement for a lightweight and affordable proof-of-concept that can be scaled or monetised in later phases without significant infrastructure burden.

See Appendix~\ref{app:budget-summary} for more information.

\section{Risk Assessment}

\begin{figure}[H]
	\centering
	\includegraphics[width=0.6\textwidth]{images/ra-table.png}
	\caption{Project risk assessment.}
\end{figure}

A comprehensive risk assessment was conducted to evaluate technical, operational, and project-execution risks that could affect system performance, data protection, or delivery timelines.
Key technical risks included API rate limits, model hallucination, cloud misconfiguration, and retrieval errors, all of which were mitigated through Identity and Access Management (IAM) reviews, billing alerts, retrieval constraints, and controlled refusal behaviour.
Operational risks---such as accidental exposure of personal data through user inputs or unsafe document uploads---were addressed using anonymised test cases, input sanitisation rules, and clear user-facing warnings.
Workplace and project risks, including dependency on external services or bottlenecks in development capacity, were moderated through modular architecture design, shared code ownership, and an iterative sprint structure.
After mitigation, all risks achieved low residual scores, providing a stable foundation for Term 8 development and deployment.

See Appendix~\ref{app:risk-assessment} for more information.

\section{Outlook}

\begin{figure}[H]
	\centering
	\begin{minipage}{0.49\textwidth}
		\centering
		\includegraphics[width=\textwidth]{images/term-7-timeline.png}
	\end{minipage}
	\begin{minipage}{0.49\textwidth}
		\centering
		\includegraphics[width=\textwidth]{images/term-8-timeline.png}
	\end{minipage}
	\caption{Outlooks for Term 7 and Term 8.}
\end{figure}

\vspacebaselineskip

Through iterative exploration in frontend, backend, cloud infrastructure, testing, and benchmarking, the team clarified the system architecture and identified the most feasible approach for a scalable compliance assistant.
Early testing also confirmed the value of retrieval-grounded responses and highlighted key areas---such as chat history design, training workflows, and analytics requirements---that will shape the next phase.
Looking ahead, Term 8 will focus on completing the core platform components: server-based chat history, PDPA document generation, the training module, and the analytics dashboard.
Development will follow the planned sprint structure, beginning with stabilising the chatbot and document flows, followed by delivering a functional training platform and a multi-metric analytics dashboard for AsiaCloud.
The final weeks will be dedicated to system hardening, evaluation, and preparing deployment-ready documentation.
Overall, the project remains on track to deliver a modular and cost-efficient proof-of-concept that supports SMEs in navigating PDPA obligations, while providing AsiaCloud with a flexible foundation for future commercial development.


\printbibliography

\appendix

\chapter{Supplementary Material}
\label{app:supplementary-material}

\section{Rational Utility}
\label{app:rational-utility}

Rational Utility, as defined by
\[
	U_\text{rational}
	=
	440A
	+
	220\left(
		\frac{20}{P+7.43}
	\right)^{0.7}
	+
	400M^{0.3}
	+
	180\left(
		\frac{1}{1+T\exp(T-3)}
	\right)
\]
decomposes into four objective components: accuracy, price, multimodality, and latency.
Each term is a transformed version of a measured quantity, scaled by a contribution constant chosen from prior work on customer intention, loyalty, or adoption.
The transformation shape and constant together determine how strongly that quantity influences the final score and where diminishing returns begin.

The first term captures accuracy.
Here $A\in\mathbb{R}_{[0,1]}$ is the empirical accuracy of a model on an evaluation dataset.
Its contribution is scaled using the largest constant in the framework, reflecting findings that accuracy is the strongest predictor of user intention and retention in service environments \parencite{alhattami2025}.
In the context of legal and compliance assistance, factual correctness is especially critical, so accuracy is treated as the dominant factor.
Full accuracy yields the maximum contribution allocated to this dimension, while partial accuracy reduces the score proportionally.

The second term captures pricing utility.
Here, $P$ is the price per million tokens of combined input and output, as listed on OpenRouter.
The functional form is decreasing in $P$: as a model becomes more expensive to run, its contribution to Rational Utility falls.
The exponent induces diminishing sensitivity, so large price differences at the low end matter more than equally large differences at the high end, consistent with behavioural findings on price elasticity.
The contribution constant is derived from reported beta sensitivities for price effects on customer concentration, and is capped such that pricing cannot contribute more utility than accuracy \parencite{tanantong2024}.
This ensures that an extremely cheap but inaccurate model does not outrank a moderately priced, highly accurate one.

The third term measures the benefit of multimodality.
Here, $M$ is the number of input modalities, as listed on OpenRouter.
Empirical work suggests that the availability of additional interaction channels modestly improves customer engagement and concentration \parencite{zhang2024}.
We encode this with a sublinear exponent, which yields strong gains from adding the first few modalities but quickly introduces diminishing returns.
The contribution constant reflects an upper bound comparable to, but slightly below, the weight of accuracy, acknowledging that modality breadth is valuable but should not overshadow factual performance.

The final term reflects latency utility.
Here, $T$ is the latency in seconds, as listed on OpenRouter.
The chosen functional form behaves similarly to a squashed inverse curve: utility is high and near-saturated for low latencies, then declines sharply as response times exceed a few seconds, and eventually flattens out for very slow models.
This shape aligns with user studies showing that delays beyond a small threshold significantly degrade perceived quality and willingness to continue using a system, while shaving off sub-second latency yields smaller marginal gains \parencite{kim2024}.
The contribution constant is calibrated from reported beta sensitivities for time-related service attributes, using the mid-range estimate as a compromise between different studies.

Together, these four terms produce a composite Rational Utility score that balances factual quality, cost, interaction richness, and responsiveness.
The relative magnitudes of the contribution constants ensure that accuracy remains primary, while still rewarding models that are affordable, multimodal, and fast enough to support practical SME workflows.

\section{Budget Summary}
\label{app:budget-summary}

We used the \href{https://calculator.aws/#/estimate?id=f3936da1c288edd9947a25383f2b1e58731e93eb}{AWS Pricing Calculator} to determine the infrastructure budget allocation.

\end{document}
