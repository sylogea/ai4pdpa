\chapter{System Implementation}

\section{Term 7}

Term 7 saw the creation of the first system implementations in two stages: Iteration 1 and Iteration 2.
Iteration 2 builds upon Iteration 1 incrementally, to ensure a manageable growth of system complexity.

In Iteration 1, we developed our first prototype: Prototype 1.

\begin{figure}[H]
	\centering
	\begin{minipage}{0.49\textwidth}
		\centering
		\includegraphics[width=\textwidth]{images/prototype-1-1.png}
	\end{minipage}
	\begin{minipage}{0.49\textwidth}
		\centering
		\includegraphics[width=\textwidth]{images/prototype-1-2.png}
	\end{minipage}
	\caption{Preview of Prototype 1.}
\end{figure}

\vspacebaselineskip

The prototype showed clear strengths.
It implemented the computational graph correctly on the backend and rendered the output of the chat model without token artefacts on the frontend.
However, the prototype also had notable shortcomings.
It did not maintain memory within conversations and spent unnecessary compute to rebuild the vector store on each run.
Overall, Prototype 1 established a functional baseline but required several improvements to support a usable workflow.

In Iteration 2, we expanded Prototype 1 with a set of core functional upgrades to develop our second prototype: Prototype 2.

\begin{figure}[H]
	\centering
	\begin{minipage}{0.49\textwidth}
		\centering
		\includegraphics[width=\textwidth]{images/prototype-2-1.png}
	\end{minipage}
	\begin{minipage}{0.49\textwidth}
		\centering
		\includegraphics[width=\textwidth]{images/prototype-2-2.png}
	\end{minipage}
	\caption{Preview of Prototype 2.}
\end{figure}

\vspacebaselineskip

Prototype 2 introduced several backend improvements.
The system ensured turns within a conversation were accumulated, by feeding past messages in the same conversation to the chat model as context on each turn.
This made the chatbot more user-friendly, as it meant the chatbot could respond using past messages in the same conversation.
The system also used a 256-Bit Secure Hash Algorithm (SHA-256) to generate a fingerprint based on information about the documents, chunking algorithm, and embedding model.
This made system startup more efficient, as fingerprints were checked before vector store building.
Prototype 2 also introduced several frontend improvements.
The chat interface produced progressive, token-by-token responses.
A consolidated account page enabled management of preferences, account details, and privacy settings.
The interface was redesigned to be responsive across devices, and a dark--light mode toggle was added.
Chats could also be exported as Portable Document Format (PDF) files, and past conversations were stored in the browser through LocalStorage.
Overall, these additions formed a more complete and usable second prototype.

\section{Term 8}

Term 8 will focus on strengthening deployment reliability and expanding the system into a more complete PDPA compliance platform.

\begin{figure}[H]
	\centering
	\includegraphics[width=0.95\textwidth]{images/planned-architecture.png}
	\caption{Planned system architecture.}
\end{figure}

\vspacebaselineskip

An automated CI/CD pipeline will be introduced using GitHub Actions.
Each push to the main branch will trigger automated tests, build container images, publish them to the registry, and update the running services on AWS App Runner.
This ensures that only validated builds are deployed and reduces manual operational overhead while maintaining consistency between the codebase and the live environment.

\begin{figure}[H]
	\centering
	\includegraphics[width=0.95\textwidth]{images/timeline.png}
	\caption{Sprint timeline for Term 8.}
\end{figure}

\vspacebaselineskip

In addition to deployment enhancements, Term 8 will extend the functional scope of the system.
The platform will support persistent server-side chat history, enabling durable and auditable interactions.
PDPA documents will be generated directly within the chat interface.
A lightweight training module will allow organisations to upload materials, generate quizzes, and provide feedback to users based on their understanding.
An analytics dashboard will surface common queries and usage patterns to support organisational insight and operational planning.
The final phase of Term 8 will focus on performance improvements, PDPA validation, and deployment refinement to produce a stable first version suitable for evaluation by AsiaCloud and SME users.
